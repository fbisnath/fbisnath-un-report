% Options for packages loaded elsewhere
\PassOptionsToPackage{unicode}{hyperref}
\PassOptionsToPackage{hyphens}{url}
%
\documentclass[
]{article}
\usepackage{amsmath,amssymb}
\usepackage{lmodern}
\usepackage{iftex}
\ifPDFTeX
  \usepackage[T1]{fontenc}
  \usepackage[utf8]{inputenc}
  \usepackage{textcomp} % provide euro and other symbols
\else % if luatex or xetex
  \usepackage{unicode-math}
  \defaultfontfeatures{Scale=MatchLowercase}
  \defaultfontfeatures[\rmfamily]{Ligatures=TeX,Scale=1}
\fi
% Use upquote if available, for straight quotes in verbatim environments
\IfFileExists{upquote.sty}{\usepackage{upquote}}{}
\IfFileExists{microtype.sty}{% use microtype if available
  \usepackage[]{microtype}
  \UseMicrotypeSet[protrusion]{basicmath} % disable protrusion for tt fonts
}{}
\makeatletter
\@ifundefined{KOMAClassName}{% if non-KOMA class
  \IfFileExists{parskip.sty}{%
    \usepackage{parskip}
  }{% else
    \setlength{\parindent}{0pt}
    \setlength{\parskip}{6pt plus 2pt minus 1pt}}
}{% if KOMA class
  \KOMAoptions{parskip=half}}
\makeatother
\usepackage{xcolor}
\usepackage[margin=1in]{geometry}
\usepackage{longtable,booktabs,array}
\usepackage{calc} % for calculating minipage widths
% Correct order of tables after \paragraph or \subparagraph
\usepackage{etoolbox}
\makeatletter
\patchcmd\longtable{\par}{\if@noskipsec\mbox{}\fi\par}{}{}
\makeatother
% Allow footnotes in longtable head/foot
\IfFileExists{footnotehyper.sty}{\usepackage{footnotehyper}}{\usepackage{footnote}}
\makesavenoteenv{longtable}
\usepackage{graphicx}
\makeatletter
\def\maxwidth{\ifdim\Gin@nat@width>\linewidth\linewidth\else\Gin@nat@width\fi}
\def\maxheight{\ifdim\Gin@nat@height>\textheight\textheight\else\Gin@nat@height\fi}
\makeatother
% Scale images if necessary, so that they will not overflow the page
% margins by default, and it is still possible to overwrite the defaults
% using explicit options in \includegraphics[width, height, ...]{}
\setkeys{Gin}{width=\maxwidth,height=\maxheight,keepaspectratio}
% Set default figure placement to htbp
\makeatletter
\def\fps@figure{htbp}
\makeatother
\setlength{\emergencystretch}{3em} % prevent overfull lines
\providecommand{\tightlist}{%
  \setlength{\itemsep}{0pt}\setlength{\parskip}{0pt}}
\setcounter{secnumdepth}{-\maxdimen} % remove section numbering
\ifLuaTeX
  \usepackage{selnolig}  % disable illegal ligatures
\fi
\IfFileExists{bookmark.sty}{\usepackage{bookmark}}{\usepackage{hyperref}}
\IfFileExists{xurl.sty}{\usepackage{xurl}}{} % add URL line breaks if available
\urlstyle{same} % disable monospaced font for URLs
\hypersetup{
  pdftitle={Writing Reports with R Markdown},
  pdfauthor={Felicia Bisnath},
  hidelinks,
  pdfcreator={LaTeX via pandoc}}

\title{Writing Reports with R Markdown}
\author{Felicia Bisnath}
\date{2022-10-21}

\begin{document}
\maketitle

This report was prepared for the UN. It analyses the relationship
between a country's GDP, life expectancy, and CO2 emissions. Our goal
was to determine to what degree a country's economic strength or
weakness may be related to its public health status and impact on
climate pollution. We hypothesise that both life expectancy and CO2
emissions will increase with a country's GDP.

\includegraphics{first_report_files/figure-latex/gdp_lifeexp_1997-1.pdf}

The above plot shows the relationship between the GDP per capita and
life expectancy for a total of 142 countries. Economic wealth ranged
from a minimum of \$ 312 to a maximum of 41283 per capita.

\begin{longtable}[]{@{}ll@{}}
\toprule()
Summary of Data & \\
\midrule()
\endhead
Number of countries & 142 \\
Minimum GDP per cap & 312 \\
Maximum GDP per cap & 41283 \\
\bottomrule()
\end{longtable}

\hypertarget{lists}{%
\section{Lists}\label{lists}}

\hypertarget{unordered-lists}{%
\subsection{Unordered lists}\label{unordered-lists}}

\begin{itemize}
\tightlist
\item
  \textbf{R}
\item
  \emph{geom}
\item
  \textbf{\emph{statistics in R}}
\end{itemize}

\hypertarget{ordered-lists}{%
\subsection{Ordered lists}\label{ordered-lists}}

\begin{enumerate}
\def\labelenumi{\arabic{enumi}.}
\item
  \textbf{R}
\item
  \emph{geom}
\item
  \textbf{\emph{statistics in R}}
\item
  \href{www.rstudio.com/resources/cheatsheets}{Cheatsheets}
\item
  \href{https://www.rstudio.com/wp-content/uploads/2015/03/rmarkdown-reference.pdf}{Markdown
  cheatsheet}
\item
  list1

  \begin{itemize}
  \tightlist
  \item
    one
  \item
    two
  \end{itemize}
\item
  list2
\end{enumerate}

```

\end{document}
